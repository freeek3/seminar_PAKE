\documentclass[peerreview, 10pt, onecolumn]{IEEEtran}
\IEEEoverridecommandlockouts
% The preceding line is only needed to identify funding in the first footnote. If that is unneeded, please comment it out.
%\usepackage{cite}
%\usepackage{amsmath,amssymb,amsfonts}
%\usepackage{algorithmic}
%\usepackage{graphicx}
%\usepackage{textcomp}
%\usepackage{xcolor}
\def\BibTeX{{\rm B\kern-.05em{\sc i\kern-.025em b}\kern-.08em
    T\kern-.1667em\lower.7ex\hbox{E}\kern-.125emX}}
\begin{document}

\title{Peer review: An Overview of DNS Cache Poisoning }

\author{}

%\maketitle
\IEEEpeerreviewmaketitle

\section*{Summary}
    The paper gives an introduction to the Domain Name System and how it is exploitable with DNS Poisoning.
    It is used to direct a user to a compromised server.
    To achieve the poisoning of the DNS entry, there are various different ways. 
    These ways are all stated and explained in the paper. 
    It also gives insight into the found solutions to the vulnerability.
    The most important solution is called DNSSEC is explained in detail.
\section*{Strength}
    \begin{itemize}
        \renewcommand\labelitemi{+}
        \item The paper gives a good overview of the DNS poison attack scheme.
        \item It lists offensive and defensive measurements.
        \item It shows that there is still work needed in adapting the defensive measurements.
        \item There is a common thread in the explanation of the attacks. 
    \end{itemize}
\section*{Weakness}
    \begin{itemize}
        \renewcommand\labelitemi{-}
        \item There are no real examples in the paper, it only lists the different attack strategies.
        \item The abstract lacks some motivation.
        \item There is no conclusion in the abstract.
        \item The paper is explaining how to flush your DNS on Windows.
        \item The predicting random number generation part is lacking an introductory part.
        \item There are often abbreviations used without an explanation.
        \item There is no Related Work
        \item There is no Background provided despite probably needed to explain the DNS System.
        \item There is no Discussion.
        \item There is no Evaluation.
        \item There is no Conclusion.
        \item There is no structure in the paper it is just a concatenation of Headings.
        \item There are no Threat Models explained in different attack strategies, suddenly there is system access.
        \item The malware part is a repetition of the bruteforce part. 
    \end{itemize}
\section*{Comments}
    \begin{itemize}
        %\renewcommand\labelitemi{-}
        \item The introduction is a good introduction to the Domain Name System, which could be moved into a background
        \item While explaining the DNS a figure would be good to understand the structure better
        \item In general more detail would be nice, how does a DNS package look like in contrast to a poisoned one.
        \item Make it more clear how DNS request is intercepted, and an attacker knows there is one being sent.
        \item Explain the "birthday paradox"
        \item Explain the UDP protocol and in the same procedure the TCP protocol. This would fit into a background.
        \item In the last paragraph of the first DNS poison attack you fail to explain why all future responses will be ignored. 
        \item Clearly describe your attacks, what does an attacker know what abilities do he has. Do not introduce the compromised servers' step by step. 
        \item Explain TTL more, again in the Background for DNS. 
        \item For the explanation of DNSSEC a figure could be helpful, as there are suddenly a lot of different new Keys introduced.
        \item At the start of the last paragraph to DNSSEC list the tasks needed to implement DNSSEC
        \item At the end of the last paragraph to DNSSEC list the attacks that come with partial deployment of DNSSEC. Make a chapter from these attacks.
        \item In the bruteforce part, there is 32-bit field used but not explained.
        \item In the end of the first paragraph of bruteforce "when" is often used. Make the Threat model more clear
        \item In the example of the bruteforce paragraph there seems to be logical error: We already have access to the target and need it to visit a malicious website to execute code on the target.
        \item The predicting the random generator seems to be a side-channel attack, which is not clearly stated.
        \item Make the threat model of the random number generator attack more clear.
        \item In the malware part, there is a focus on Windows. There are other operating systems out there.
        \item Evaluate the current state of DNS make a conclusion if the attacks are still feasible.
        \item More drawn out examples would be helpful to understand the topic more easily. 
        \item When do we know that a victim has sent a DNS request?
    \end{itemize}

\end{document}
