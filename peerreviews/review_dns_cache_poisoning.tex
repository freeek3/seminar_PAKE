\documentclass[peerreview, 10pt, onecolumn]{IEEEtran}
\IEEEoverridecommandlockouts
% The preceding line is only needed to identify funding in the first footnote. If that is unneeded, please comment it out.
%\usepackage{cite}
%\usepackage{amsmath,amssymb,amsfonts}
%\usepackage{algorithmic}
%\usepackage{graphicx}
%\usepackage{textcomp}
%\usepackage{xcolor}
\def\BibTeX{{\rm B\kern-.05em{\sc i\kern-.025em b}\kern-.08em
    T\kern-.1667em\lower.7ex\hbox{E}\kern-.125emX}}
\begin{document}

\title{Peer review: An Overview of DNS Cache Poisoning }

\author{}

%\maketitle
\IEEEpeerreviewmaketitle

\section*{Summary}
    The paper gives an introduction to the Domain Name System and how it is exploitable with DNS Poisoning.
    It is used to direct a user to a compromised server.
    To achieve the poisoning of the DNS entry, there are various different ways. 
    These ways are all stated and explained in the paper 
    It also gives insight into the found solutions to the vulnerability. 
    The most important solution is called DNSSEC is explained in detail.
\section*{Strength}
    \begin{itemize}
        \renewcommand\labelitemi{+}
        \item The paper gives a good overview of the DNS poison attack scheme.
        \item It lists offensive and defensive measurements
        \item It shows that there is still work needed in adapting the defensive measurements.
        \item There is a common thread in the explanation of the attacks 
    \end{itemize}
\section*{Weakness}
    \begin{itemize}
        \renewcommand\labelitemi{-}
        \item test
        \item 
    \end{itemize}
\section*{Comments}
    \begin{itemize}
        %\renewcommand\labelitemi{-}
        \item Test
        \item 
    \end{itemize}

\end{document}
